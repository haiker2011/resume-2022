%-------------------------------------------------------------------------------
%	SECTION TITLE
%-------------------------------------------------------------------------------
\cvsection{开源项目}


%-------------------------------------------------------------------------------
%	CONTENT
%-------------------------------------------------------------------------------
\begin{cventries}

  \cventry
    {} % Job title
    {\href{https://github.com/karmada-io/karmada}{karmada.io}} % Organization
    {GitHub} % Location
    {2022 年 2 月} % Date(s)
    {
      \begin{cvitems} % Description(s) of tasks/responsibilities
        \item karmada 部分功能的开发与重构,后续KingAI支持集成 karmada,对多集群管理调度提供支持。
      \end{cvitems}
    }

    \cventry
    {} % Job title
    {\href{https://github.com/kubeedge/sedna}{kubeedge.io}} % Organization
    {GitHub} % Location
    {2022 年 2 月} % Date(s)
    {
      \begin{cvitems} % Description(s) of tasks/responsibilities
        \item 参与 sedna 开源项目,新增同步配置、PR模板等功能。
      \end{cvitems}
    }

    \cventry
    {} % Job title
    {\href{https://github.com/kubeedge/kubeedge}{kubeedge.io}} % Organization
    {GitHub} % Location
    {2022 年 2 月} % Date(s)
    {
      \begin{cvitems} % Description(s) of tasks/responsibilities
        \item 参与 kubeedge 开源项目,扩展helm配置,提供更加便利的配置选项等。
      \end{cvitems}
    }

    \cventry
    {} % Job title
    {\href{https://github.com/kubesphere/kubekey}{kubesphere.io}} % Organization
    {GitHub} % Location
    {2022 年 2 月} % Date(s)
    {
      \begin{cvitems} % Description(s) of tasks/responsibilities
        \item kubekey 部分功能的重构,当前正在参与 kubesphere 与 karmada 对接调研,kubesphere 提供多集群调度功能。
      \end{cvitems}
    }

  \cventry
    {} % Job title
    {\href{https://github.com/istio/istio.io}{istio.io}} % Organization
    {GitHub} % Location
    {2019 年 6 月} % Date(s)
    {
      \begin{cvitems} % Description(s) of tasks/responsibilities
        \item istio 官方文档中文化维护,翻译 istio release 1.1 更新文档
      \end{cvitems}
    }

  \cventry
    {} % Job title
    {\href{haiker2011/awesome-nlp-sentiment-analysis}{awesome-nlp-sentiment-analysis}} % Organization
    {GitHub} % Location
    {2022 年 2 月} % Date(s)
    {
      \begin{cvitems} % Description(s) of tasks/responsibilities
        \item 收集NLP领域相关的数据集、论文、开源实现,尤其是情感分析、情绪原因识别、评价对象和评价词抽取方面,461 stars
      \end{cvitems}
    }

  \cventry
    {} % Job title
    {\href{https://github.com/servicemesher/getting-started-with-knative}{getting-started-with-knative}} % Organization
    {GitHub} % Location
    {2022 年 2 月} % Date(s)
    {
      \begin{cvitems} % Description(s) of tasks/responsibilities
        \item 《Knative 入门中文版》第三章翻译,其他章节 review,224 stars,将由 Pivotal 公司印刷出版
      \end{cvitems}
    }

%   \cventry
%     {} % Job title
%     {\href{https://github.com/coala/coala-vs-code}{coala-vs-code}} % Organization
%     {GitHub} % Location
%     {2017 年 2 月} % Date(s)
%     {
%       \begin{cvitems} % Description(s) of tasks/responsibilities
%         \item 为 coala 社区实现的,基于 Language Server Protocol 的 VS Code 插件,21 stars
%       \end{cvitems}
%     }

%   \cventry
%     {} % Job title
%     {\href{https://github.com/dyweb/scrala}{Scrala}} % Organization
%     {GitHub} % Location
%     {2016 年 2 月} % Date(s)
%     {
%       \begin{cvitems} % Description(s) of tasks/responsibilities
%         \item 基于 Scala Actor 模型,受 scrapy 启发的爬虫框架,100 stars
%       \end{cvitems}
%     }


\end{cventries}