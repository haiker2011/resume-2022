%-------------------------------------------------------------------------------
%	SECTION TITLE
%-------------------------------------------------------------------------------
\cvsection{个人项目}


%-------------------------------------------------------------------------------
%	CONTENT
%-------------------------------------------------------------------------------
\begin{cventries}

  \cventry
    {} % Job title
    {\href{https://github.com/siglt/tosknight}{Tosknight}} % Organization
    {GitHub} % Location
    {2018 年 1 月} % Date(s)
    {
      \begin{cvitems} % Description(s) of tasks/responsibilities
        \item 定期记录各个网站用户协议或隐私条款版本变化的工具,20 stars
      \end{cvitems}
    }

  \cventry
    {} % Job title
    {\href{https://github.com/prism-river/killy}{Killy}} % Organization
    {GitHub} % Location
    {2017 年 10 月} % Date(s)
    {
      \begin{cvitems} % Description(s) of tasks/responsibilities
        \item Minecraft 插件,在 Minecraft 中以游戏的方式运维 TiDB 集群,Hackathon 作品,51 stars
      \end{cvitems}
    }

  \cventry
    {} % Job title
    {\href{https://github.com/gaocegege/maintainer}{Maintainer}} % Organization
    {GitHub} % Location
    {2017 年 5 月} % Date(s)
    {
      \begin{cvitems} % Description(s) of tasks/responsibilities
        \item 帮助开发者更好地维护其 GitHub 上仓库的命令行工具,128 stars
      \end{cvitems}
    }

  \cventry
    {} % Job title
    {\href{https://github.com/coala/coala-vs-code}{coala-vs-code}} % Organization
    {GitHub} % Location
    {2017 年 2 月} % Date(s)
    {
      \begin{cvitems} % Description(s) of tasks/responsibilities
        \item 为 coala 社区实现的,基于 Language Server Protocol 的 VS Code 插件,21 stars
      \end{cvitems}
    }

  \cventry
    {} % Job title
    {\href{https://github.com/dyweb/scrala}{Scrala}} % Organization
    {GitHub} % Location
    {2016 年 2 月} % Date(s)
    {
      \begin{cvitems} % Description(s) of tasks/responsibilities
        \item 基于 Scala Actor 模型,受 scrapy 启发的爬虫框架,100 stars
      \end{cvitems}
    }


\end{cventries}
