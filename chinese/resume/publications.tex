%-------------------------------------------------------------------------------
%	SECTION TITLE
%-------------------------------------------------------------------------------
\cvsection{学术论文、专利}

%-------------------------------------------------------------------------------
%	CONTENT
%-------------------------------------------------------------------------------
\begin{cventries}

% %---------------------------------------------------------
%   \cventry
%     {Ce Gao, Rui Ren and Hongming Cai} % Award
%     {GAI: A Centralized Tree-Based Scheduler for Machine Learning Workload in Large Shared Cluster} % Event
%     {ICA3PP'18 (CCF-C)} % Location
%     {2018.11} % Date(s)
%     {
%       \begin{cvitems} % Description(s)
%         \item {
%           本文分析了机器学习模型的训练,识别了训练过程中的短板效应:与 CPU 训练相比,GPU 训练需要更高的网络带宽。这一观察启发了 GAI 的设计,GAI 是一个集中式的调度器,用于机器学习工作负载。它依赖于两种技术:1)树型结构。该结构分层存储集群信息,实现多层调度。2)扩展良好的优先级算法。我们全面考虑了模型培训工作的多个优先级,以支持资源退化和抢占。在 Kubernetes、Kubeflow 和 TensorFlow 上实现了 GAI 的原型。它是通过一个模拟器和一个真正的基于云的集群进行评估的。结果表明,在 DL 模型上,调度吞吐量提高了28\%,训练收敛速度提高了21\%
%         }
%       \end{cvitems}
%     }


%---------------------------------------------------------
\cventry
{孙海洲} % Award
{一种云服务的重启方法和装置} % Event
{2021102877380} % Location
{2021.3} % Date(s)
{
  \begin{cvitems} % Description(s)
    \item {
      本专利提出使用 Kubernetes 来在不重启服务的情况下更新配置文件。
    }
  \end{cvitems}
}

%---------------------------------------------------------
  \cventry
    {孙海洲} % Award
    {登录请求的验证方法及装置、存储介质、电子设备} % Event
    {2021102871967} % Location
    {2021.3} % Date(s)
    {
      \begin{cvitems} % Description(s)
        \item {
          本专利提出一种新的登陆AI容器的方法,可以提高安全隔离性。
        }
      \end{cvitems}
    }

%---------------------------------------------------------
\cventry
{张凯, 孙海洲, 俞晓明, 刘悦, 程学旗} % Award
{一种基于 Kubernetes 的 JS 解析方法及系统} % Event
{201910578638.6} % Location
{2019.6} % Date(s)
{
  \begin{cvitems} % Description(s)
    \item {
      本专利提出使用 Kubernetes 来提高服务平台资源利用率,根据不同优先级不同需求动态调整任务。
    }
  \end{cvitems}
}

\end{cventries}
