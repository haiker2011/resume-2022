%-------------------------------------------------------------------------------
%	SECTION TITLE
%-------------------------------------------------------------------------------
\cvsection{工作经历}


%-------------------------------------------------------------------------------
%	CONTENT
%-------------------------------------------------------------------------------
\begin{cventries}

%---------------------------------------------------------
\cventry
{机器学习平台研发工程师} % Job title
{北京金山云网络技术有限公司} % Organization
{中国北京} % Location
{2020 年 7 月至今} % Date(s)
{
  \begin{cvitems} % Description(s) of tasks/responsibilities
    \item KingAI机器学习平台—— AI 容器开发模块研发,主要包含ssh登陆、jupyter notebook集成与鉴权、分布式开发与训练,打通整个KingAI
    平台模块,提供一站式的机器学习开发体验。
    \item KingAI机器学习平台——推理服务研发,重新设计规划推理服务模块,重构在线推理,大幅精简在线推理服务数量,提高推理服务稳定性。
    \item KingAI机器学习平台——可视化建模研发,利用 Argo 实现单步运行,提高可视化建模运行速度。
    \item KingAI机器学习平台——服务细粒度鉴权、服务流量治理等。
  \end{cvitems}
}

%---------------------------------------------------------
  \cventry
    {学生} % Job title
    {中国科学院计算技术研究所} % Organization
    {中国北京} % Location
    {2017 年 8 月至 2020 年 7 月} % Date(s)
    {
      \begin{cvitems} % Description(s) of tasks/responsibilities
        \item 关键字、情感分析、命名实体识别、分词、文本分类、新闻抽取等分布式服务框架重构和新版开发,重构一套框架出来,通过插件机制,显著提高服务扩展性。
        \item 搭建持续集成Jenkins + Harbor + Kubernetes 集群,完成服务持续集成、持续发布,加速交付,完成了实验室从代码开发到部署测试环境自动化,提高开发效率。
        \item 分布式服务框架云原生化,引入 Istio 和 Knative,进一步提高服务治理,引入服务的监控、可视化追踪,减轻从测试环境到生产环境中开发和运维人员负担。
      \end{cvitems}
    }

%---------------------------------------------------------
  \cventry
    {Android 开发工程师} % Job title
    {北京数字认证股份有限公司} % Organization
    {中国北京} % Location
    {2015 年 7 月至 2017 年 7 月} % Date(s)
    {
      \begin{cvitems} % Description(s) of tasks/responsibilities
        \item 产品开发:负责 Android 端产品 SDK 需求定义与开发
        \item 项目定制:中国人寿、上海银行等项目定制开发
        \item 项目实施:负责项目使用 SDK 实施的技术目持
      \end{cvitems}
    }

\end{cventries}
